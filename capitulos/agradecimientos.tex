\chapter{Agradecimientos} \label{cap:agradecimientos}
Queremos agradecer en primer lugar a nuestras familias, parejas, amigos y compañeros por su gran apoyo e interés durante la realización de este proyecto.

Además, estamos especialmente agradecidos por acogernos como tutores, así como ayudarnos con su trabajo, su paciencia y su dedicación a Carlos Cervigon Rückauer y a Antonio A. Sánchez-Ruiz, que nos han orientado tanto a la hora de realizar y estructurar el Trabajo de Fin de Grado de forma que fuera lo menos caótico posible y llegase a buen puerto, a costa de infinitas revisiones, correcciones y tiempo.
 
También agradecerles su trabajo y dedicación a Philipp Rohlfshagen, Simon Lucas y David Robles, creadores del fantástico \textit{framework} que usamos para Pac-Man, con los que hemos tenido oportunidad de establecer contacto. Es una suerte disponer de proyectos como el suyo, que aún a día de hoy sigue generando competiciones que motivan a gente como nosotros a investigar y desarrollar.
 
Por supuesto no podemos olvidar a José Luis Risco Martín, José Manuel Colmenar Verdugo y Josué Pagán Ortiz, creadores y desarrolladores del \textit{framework}  JECO, sin el cual no habría sido posible este trabajo.
 
Y por último y no menos importante, queremos expresar nuestro agradecimiento a la Universidad Complutense de Madrid, a la Facultad de Informática y en especial a sus profesores, gracias a los cuales hemos tenido la oportunidad de aprender las bases necesarias para llegar hasta aquí y poder realizar un trabajo en el que hemos tenido la oportunidad y la necesidad de entremezclar conocimientos de un montón de facetas de la informática.
